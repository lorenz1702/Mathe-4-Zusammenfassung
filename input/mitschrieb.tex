\section{Mitschrieb}


\subsection{05.04}

Themenüberblick
\begin{itemize}
    \item Deskriptive Statistik
    \item Merkmalräume und Ereignisse
    \item Wahrscheinlichkeitsräume 
    \item Kombinatorik 
    \item Zufallsvariablen
\end{itemize}

\textbf{Deskriptive Statistik} ist Wahrscheinlichkeitstheorie. 
Es gibt Regeln für Zufallsexperimente. \\

\textbf{absolute/relative Häufigkeit}\\
$H_x(a_j)$:= Anzahl der in der Stichprobe $x$ vorkommenden Stichprobenelement

\subsection{03.05}
seite 21,22 sind nicht klausurrelevant
aufgabe 12,13,14 nicht klausurrelevant 

\subsection{10.05}


\subsection{31.05}

\textbf{Die Aufgaben wurden am Anfang besprochen.}

Disjunkt heißt das sich zwei wahrscheinlichkeiten gegenseitig ausschließen. Siehe Lösungsvorschlag 7L.\\
Wahrscheinlichkeiten verunden erstellt man durch Multiplikation.
\[P(O\cap M3)=P(O|M3)\cdot P(M3)\]
Verodern erstellt man in dem man die Wahrscheinlichkeit der beiden Ereigniss addiert und mit der Wahrscheinlichkeit der verundung der beiden ereignisse subtrahiert.
\[P(O\cup M3)=P(O)+P(M3)-P(O\cap M3)\]
Aufgabe 18 ist ein gute Übung für bedingte wahrscheinlichkeit.
Antworten müssen genau hinzuschrieben werden nicht einfahc raten auch wenn es vielleicht richtig ist.\\
Aufgabe 20a als Beispiel in die Zusammenfassung. Bedingte Wahrscheinlichkeit!!!!\\
Aufgabe 20b ist auch sehr interresant.\\
\textbf{Erwartungswert, Varianz, Standardabweichung}\\

\textbf{Erwartungswert}
$X(w_j):$ Gewinn bei Ausgang $w_j$\\
$n$ Spiele, $h_j$ mal triet der Ausgang $w_j$ auf.
Frage: Wie kommt man auf $h_j$?\\
Man macht ein paar versuche wenn $n$ groß genug ist kann man die Wahrscheinlichkeit näherungsweise bestimmen.
\[X(w_1)\cdot \underbrace{\frac{h_1}{n}}_{\rightsquigarrow p(w_1)}\]
Tabelle auf Seite 12 ist wichtig für den Erwartungswert sinnvoll in die Zusammenfassung zu machen.
Frage: Was heißt diskrekt und nicht diskrekt?\\
Was kann mal alles mitnehmen in die Prüfung?\\
Seite 24,25,26 ist nicht klausurrelevant.


