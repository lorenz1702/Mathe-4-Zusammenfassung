\section{Konfidenzintervall}
Beschreibt die Genauigkeit eines Schätzwertes mit einer Wahrscheinlichkeit von mindestens $\gamma$(Konfidenzniveau) enthält geschätztes Intervall den wahren Werte
\begin{align*}
    KI &= \overline{x} \pm \overbrace{Z}^{\textrm{Z-Wert für das Konfidenzniveau}} \cdot \frac{\overbrace{\sigma}^{\textrm{Standardabweichung}}}{\underbrace{\sqrt{n}}_{\textrm{Stichprobengröße}} }\\
    \mu& \in \left[\overline{x}-Z\cdot \frac{\sigma}{\sqrt{n}},\overline{x}+Z\cdot \frac{\sigma}{\sqrt{n}}\right] \\
    Z &=\frac{\overline{X}-\mu}{\frac{\sigma}{\sqrt{n}}},\textrm{Wird aber meinst aus der Tabelle abgelesen} 
\end{align*}
Bestimmen des \textbf{symmetrische Konfidenzintervall} \[q:=\frac{1+\gamma}{2}\]
Bestimme das \textbf{einseitige Konfidenzintervall} mit oberer Schranke welches nach \textbf{unten} offen ist.
\[P(Z\geq q)= \gamma \Leftrightarrow 1-\Phi (q)= \gamma \Leftrightarrow \Phi (-q) =0.95 \overbrace{\Leftrightarrow}^{\textrm{Tabelle}}-q = 1.65 \Leftrightarrow q = -1.65 \]
\[(-\infty,\overline{x}+q\frac{\sigma}{\sqrt{n}})\]

\textbf{Sigmaumgebung mit einer festen prozentualen Richtgröße}
\begin{align*}
    P&(\mu -1.64 \sigma \leq x\leq \mu+1.64\sigma) = 90\%\\
    40&\%\textrm{ mögen Mathe}, n=100, \mu = 40, \sigma = 4.9\\
    &[31.96; 48.04]
\end{align*}
Darin liegen 90\% aller Möglichkeiten das sie sagen das sie Mathe mögen\\
\textbf{Vertrauensintervall}
\begin{align*}
    400 &= n; 120 \textrm{ sagen Mathe ist Ok}\\
    &\textrm{Relative Trefferhäufigkeit}: h= \frac{120}{400} = 0.3\\
    90\%&:\left[ h-1.64\sqrt{\frac{h\cdot(1-h)}{n}};h+ 1.64\sqrt{\frac{h\cdot(1-h)}{n}}\right]\\
    &[0.26;0.34]
\end{align*}
In welchem Bereich die Wahrscheinlichkeit liegt das jemand Matht mag.

