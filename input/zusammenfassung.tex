

\section{Wichtige diskrete Verteilungen}

\subsection{Binomialverteilung}

Seien $n$ eine natürliche Zahl und $0\leq p\leq 1$. Die zur Zähldichte

\[k\longrightarrow f(k)=\binom{n}{k}\cdot p^k\cdot (1-p)^{n-k},k=0,...,n, \]

gehörende Verteilung heißt \texttt{Binomialverteilung} mit den Parametern $n$ und $p$ und wird mit $Bin(n,p)$ bezeichnet.\\

\textbf{Beispiel: } Der Tetraeder wird 4 Mal geworfen. Berechnen sie die Wahrscheinlichkeiten der Ereignisse:
\begin{itemize}
    \item Jedes Mal die \dq 1\dq $=\binom{4}{4} \cdot (\frac{1}{4})^4\cdot (1-\frac{1}{4})^{0}\approx 0.0039$
    \item Zweimal die \dq 1\dq $=\binom{4}{2} \cdot (\frac{1}{4})^2\cdot (1-\frac{1}{4})^{4-2}\approx 0.2109$
    \item Alle Augenzahlenverschieden $=\frac{4}{4}\cdot\frac{3}{4}\cdot \frac{2}{4}\cdot \frac{1}{4} \approx 0.09375$
\end{itemize}


\subsection{Poisson-Verteilung}

Ideales Zufallsexperiment mit kleiner Erfolgswahrscheinlichkeit $p$ werde $n$ mal in unabhängiger Folge durchführt ($n$ groß). Ist $X$ die zufällige Anzahl von Treffern, so gilt mit $\lambda := n\cdot p$ näherungsweise

\[\mathbb{P} (X=k) = \binom{n}{k}\cdot p^k \cdot (1-p)^{n-k}\thickapprox e^{-\lambda}\cdot \frac{\lambda^k}{k!}  \]

für $k\leq n$.

Wann es angewendet wird muss sic noc anescaut werden.

