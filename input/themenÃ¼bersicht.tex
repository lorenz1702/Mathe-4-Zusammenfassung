\section{Themenübersicht}
Die Themenübersicht ist temporär bis sich das in halts verzeichnis selbst erzeugt
Rot heißt klausurrelevant
\begin{itemize}
    \item Wahrscheinlichkeitsräume Wahrscheinlichkeitsverteilung, Laplace-Modell Satz 3.2
    \item Kombinatorik
    \begin{itemize}
        \item Multiplikationsregel
        \item Summenregel
        \item k-Permutation \& k-Kombination mit / ohne Wiederholung
    \end{itemize}
    \item Zähldichte
    \item Wichtige diskrete Verteilungen
    \begin{itemize}
        \item Binomialvertelung
        \item Poission-Verteilung
    \end{itemize}
    \item Wichtige stetige Verteilungen
    \begin{itemize}
        \item Exponential-Verteilung
        \item Normalverteilung
    \end{itemize}
    \item Übergangswahrscheinlichkeiten und bedingte Wahrscheinlichkeitsverteilung
    \begin{itemize}
        \item Bedingte Wahrscheinlichkeits
        \item Formel von der totalen Wahrscheinlichkeit
        \item Formel von Bayes
    \end{itemize}
    \item Maßzahlen von Verteilungen
    \begin{itemize}
        \item Erwartungswert
        \item Varianz
        \item Standardabweichung Satz 12.3 Satz 12.7
    \end{itemize}
    \item Parameterschätzung Konfidenzintervall
\end{itemize}


Kurz vor den Prüfungen
\begin{itemize}
    \item Wahrscheinlichkeitsraum
    \item Binomialverteilung
    \item Possion-Verteilung
    \item Exponential-Verteilung
    \item Konfidenzintervall
    \item 
\end{itemize}
\subsection{Zusammenfassung vor den } 
1 Vorlesung ist nicht so wichtig für die Prüfung
2 Vorlesung Wie wird die verundung berechent wenn sie Stochastisch abhängig sind?
Aufgabe 3 ist nicht klausurrelevant 
Aufgabe 4 ist wichtig 
Aufgabe 5 eher nicht 
Aufgabe 6 ist nicht so wichtig aber gute übung 
Aufgabe 7 ist wichtig 
3 Vorlesung Wichtig ist konkrete Zufallsvariable (Diskrete Zufallsvariable) 
Zähldichte ist interresant wenn müssen alle ereignisse für B Addieren
Ab seite 30 nicht mehr wichtig
Aufgabe 8 ist sehr wichtig Laplace-Experminente
Aufgabe 9 nicht wirklich wichtig
4 Vorlesung Binomialverteilung und Poissionverteiliung ist sehr wichtig

Aufgabe 10 wichtig
Aufgabe 11 nicht wichtig
5 Vorlesung
Aufgabe 12 nicht
Aufgabe 13 nicht
Nur Aufgabe 15 ist wichtig

6 Vorlesung
Aufgabe 16 sehr wichtig
Aufgabe 17 ist wichtig

7 Vorlesung Gedechnislosigkeit kann verwendetet werden 

Aufgabe 18 wichtig
Aufgabe 19 nicht interresant
AUfgabe 20 wichtig

Aufgabe 21 ist wichtig
Aufgabe 23 ist wichtig

9 Vorlesung

Ab seite 13
Auf seite 20 steht wie die intervall geöffnet sein können und es ist wie bei Daniel Jung !!

Bis 22 ist es wichtig

Seite 23 ist nicht wichtig

Aufgabe 24 ist Sinnvoll für das verständnis

Aufgabe 25 ist sehr wichtig



Seite 