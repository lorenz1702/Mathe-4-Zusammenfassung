\section{Kombinatorik}
Erscheint ein Laplace-Modell in einer Situation angemessen, so ist die Wahrscheinlichkeit eines Ereignisses $A$

\[P(A)=\frac{\textrm{Anzal der für $A$ \dq günstigen \dq Fälle}}{\textrm{Anzal aller mölicen Fälle}} \]

Die Lere vom systematischen Abzlen endlicher \dq strukturierter \dq Mengen heißt Kombinatorik.

\subsection{Multiplikationsregel (Produktregel)}

\[m_1\cdot m_2 \cdot ... \cdot m_k\]

Beispiele: Wie viele Möglichkeiten ggibt es jeweils?\\

\begin{itemize}
    \item Auf der Speisekarte stehen 3 Vorspeisen, 10 Hauptgerichte und 2 Desserts. Wie viele 3-gängige Menüs kann man zusammenstellen? $\Longrightarrow 3\cdot 10\cdot 2 = 60$
    \item Ein roter und ein blauer Würfel werden geworfen.$\Longrightarrow 6\cdot 6=36$
    \item Bei einem Neuwagen gibt es 8 aufpriespflichtige Zusatzoptionen, die in beliebiger Kombination zu- oder abbestellt werden können.$\Longrightarrow 2\cdot 2\cdot ...\cdot 2 = 2^8=256$
\end{itemize}

Wenn es Experminente sind in denen zurückgelegt wird kann die \texttt{Produktregel nicht angewendet} werden, da Werte nicht beliebig kombinierbar sind.


\subsection{Summenregel}

Wenn es mehrere Gruppen mit $m_1$ bzw. $...m_n$ verschiedenen Werte gibt, die sich gegenseitig ausschließen, dann gibt es insgesamt $m_1+...+m_n$ mögliche Werte.\\

\texttt{Beispiel}\\
Auf der Speisekarte sind als Hauptspeise 10 Fleischgerichte, 4 Fischgerichte und 2 vegetarische Gerichte zur Auswahl. Zusätzlich gibt es 3 Vorspeisen und 5 Desserts zur Auswahl. Wie viele 3 gängige Menüs kann man zusammenstellen?

\[3\cdot (10+4+2)\cdot 5=240\]

\subsection{Permutation u. Kombinationen}

Tabelle wird nachgereicht solbald internet\\

Hier ist:

\begin{align}
    \binom{m}{l}&:=\frac{m!}{l!\cdot (m-l)!} (m,l\in\mathbb{N}_0,l\leq m)\\
    0!&=1\\
    m!&=1\cdot 2\cdot ...\cdot m.\\
    n^{\underline{k}}&=\frac{n!}{(n-k)!}.  
\end{align}

k-Permutation aus M ohne Wiederholung

\[\Longrightarrow \left\lvert \Omega \right\rvert = \frac{n!}{(n-k!)}=\binom{n}{k}k!\]

\[\left[ \binom{n}{k}=\frac{n!}{k!(n-k)!}\right]\]

k-Kombination aus M ohne Wiederholung

\[\Longrightarrow \left\lvert \Omega \right\rvert = \frac{n!}{k!(n-k!)}=\binom{n}{k}\]

k-Kombination aus M mit Wiederholung

\[\Longrightarrow \left\lvert \Omega \right\rvert = \binom{n+k-1}{k}\]


\begin{table}[ht]
    \begin{tabular}{|l|l|l|l|}
    \hline
    \textbf{Wdh?} & \textbf{Rhf?} & \textbf{Anzahl. Möglichkeiten} & \textbf{Beispiel} \\ \hline
    mit\footnote[1]{Mehrmals die gleiche Wahhl treffen (\textbf{mit Wdh.})}           & mit\footnote[3]{Es ist wichtig, in welcher Reihenfolge wie endschieden wurde(\textbf{mit Reihenf.})}           & $n^k$                        & $k$ Personen werfen je einen Würfel(also $n=6$)           \\ \hline
    ohne\footnote[2]{Jedes Mal eine andere Wahl treffen muss(\textbf{o. Wdh.})}         & mit\footnotemark[3]          & $n!$                    & Auf wie vielen Arten lassen sich $n$ Objekte sortieren?     \\ \hline
    ohne\footnotemark[2]         & mit\footnotemark[3]           & $\frac{n!}{(n-k)!} $                    & \begin{tabular}[c]{@{}l@{}}Lotto  \dq$k$ aus $n$\dq mit Ziehungs-Reihenfolge\\.\end{tabular}            \\ \hline
    ohne\footnotemark[2]         & ohne\footnote[4]{Es ist nur wichtig, wie oft welche Entscheidung getroffen wurde.\textbf{o. Reihenf.}}          & $\binom{n}{k}=\frac{n!}{k!\cdot(n-k)!} $                         & \begin{tabular}[c]{@{}l@{}}Normales Lotto \dq$k$ aus $n$\dq \\.\end{tabular}          \\ \hline
    mit\footnotemark[1]           & ohne\footnotemark[4]          &  $\binom{n+k-1}{k}$                               &  \begin{tabular}[c]{@{}l@{}}$k$ nicht unterscheidbare Würfel\\ werden in einem Würfelbecher geworfen  \end{tabular}        \\ \hline
    \end{tabular}
    \caption{Wann muss welche Formel angewendet werden?}
    \label{tab:my-table}
    \end{table}
